\documentclass[a4paper,11pt]{article}
\usepackage{f420}

\begin{document}

\section{The coin problem}

A detailed introduction into amortized analysis, including
the definition of potential function and examples of estimates of it
is given in~[CLRS].
Here we focus on one classical result concerning amortized analysis.
It is rather folklore and is given as an exercise in several university courses;
we show it here to establish its formal proof, because it is crucial for
understanding of what we prove further.

Given $n$ coins stored in $n$ piles (some piles may be empty). A series of operations
is executed on these piles. One operation consists of selecting a pile and distributing
all its coins equally among other piles, one coin per pile. Piles that receive a coin
and piles that do not can be chosen freely. An example of such operations can be
seen in Figure~\ref{fig:coins-distrib}. We assume the number of piles is equal
to the number of coins so that these operations can be executed
on any configuration of coins.

\input{img/coins-distrib}

Clearly during one operation at most $n$ coins are distributed. However, during $k$ consecutive operations the average number of coins disrtibuted per operation is $o(n)$. Informally, each operation makes the heights of the piles more uniform, making it impossible to have a large pile to distribute after each of the operations. This observation is formalized by the following theorem:

\begin{theorem} \label{thm:coins}
	For $k \ge \sqrt{n}$, if $k$ consecutive operations are executed, then the average number of coins distributed per operation is at most $3\sqrt{n}$.
\end{theorem}

\begin{proof} Denote piles by $p_1, \ldots, p_n$. We introduce a potential function that describes the configuration of piles and helps estimate the number of coins distributed during an operation executed on a pile:
	\[ \Phi = \sum\limits_{j=1}^n \min \left\{ \text{size}(p_j), \sqrt{n} \right\}. \]

Note that $\Phi$ is always at most $n$, since $n$ is the sum of actual sizes of all the piles. Denote by $T_i$ the number of coins distributed during the $i$-th operation, and by $\Phi_i$ the value of the potential after the $i$-th operation. Therefore, $\Phi_{0}$ and $\Phi_{k}$ are the values of the potential before and after the execution of all the operations respectively.

We will now estimate $T_i + \Phi_{i-1} - \Phi_i$. To do so, note that the pile that is being distributed (without loss of generality, $p_1$) decreases in size to zero, and several other piles increase in size by~1. Distributing $p_1$ makes $\Phi$ decrease by at most $\sqrt{n}$, regardless of $\text{size} (p_1)$ being greater or less than~$\sqrt{n}$, by definition of $\Phi$. Figure~\ref{fig:pot-affects} illustrates it: a change in size of a pile does not affect the potential if the size of the pile is greater than~$\sqrt{n}$.

\input{img/pot-affects}

Several other piles receive one coin, the number of such piles is equal to $T_i$. However, some of these piles may not contribute to the change in the potential, again because their size is greater than $\sqrt{n}$. Note that the number of such large piles is at most $\sqrt{n}$. Thus, when the coins are being put into piles, there is at least $T_i - \sqrt{n}$ increase in potential. Combining these observations,
	\[ T_i + \Phi_{i-1} - \Phi_i\ \le\ 
	   T_i + \sqrt{n} - \ll T_i - \sqrt{n} \rr\ =\ 2 \sqrt{n}. \]

We can now sum up $T_i + \Phi_{i-1} - \Phi_i$ for each of the consecutive operations. Note that $\Phi_0 - \Phi_k$ is between $-n$ and $n$.
	\[ \sum\limits_{i=1}^k \ll T_i + \Phi_{i-1} - \Phi_i \rr\ =\ 
	   \sum\limits_{i=1}^k T_i + \Phi_0 - \Phi_{k}\ \le\ 2\sqrt{n} \cdot k + n. \]

This means average $T_i$ per operation is at most $2\sqrt{n} + \frac{n}{k} \le 3\sqrt{n}$. \end{proof}

Informally, this theorem means that if during an operation one pile
loses many coins, and many distinct piles get at most one coin each,
there can only be so much of such operations.

\end{document}
